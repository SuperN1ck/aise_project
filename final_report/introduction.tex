\chapter{Introduction}

The amount of information is exponentially increasing, a robust and delicate index structure also be needed. SQL, which is a language to manage Database Management System(DBMS) is widely used as a tool which arranges effectively indices and relationship between tables, the area of application is very large, such as Electronic Health Record(EHR), customer management, and distribution management. Software developers can get necessary information through query statement in SQL. However, they need to verify their own query statement and check whether unexpected bugs exist. Otherwise, we can't believe the results of a query and beyond lose reliability of SQL. After verifying, we can judge whether this query statement is semantic or not. If the query statement is simple, developers can make test cases manually considering coverage targets. However, if a query statement is complex, this manual making is very inefficient and hard to deal with.

To handle this problem, EvoSQL\cite{castelein2018search} made test case generation tool covering whole coverage targets for SQL queries using a genetic algorithm(GA). It shows astonishingly covering performance, 98.6\%. Their contributions are, first suggesting the definition of test case generation problem for SQL and a ``physical query plan'', which is a series of step to solving in a query. Through the physical query plan, they defined a fitness function and implemented crossover and mutation as general GA. They presented seeding strategies by generating a seeding pool for initial population, thus helps reaching a specific strings or integer including date. Finally,  with comparison of three methods (Random search, Biased Random Search (with Seeding strategy), GA), they showed how GA covers coverage targets of various queries.
 

GA performs well than the other two methods. But we thought of any other improvements than GA, and wondered why they don't we approach differently, using Multi-Objective Optimization(MOO) methods which is widely used in solving software engineering problems. We assumed the weakness of GA in EvoSQL is that they don't arrange the order of coverage targets to solve, so that we should wait for solving easy coverage target if it is behind difficult coverage targets. Also, coverage targets don't share their semantic discovery to others because of limitation on single target strategy.
 

To apply MOO, we expected that our model can be guided by a similar solution from other coverage targets. Therefore, we first applied NSGA-II\cite{deb2002fast} as a basic MOO method, changed the crowding distance into ``sort fronts by covered target'' and added combine operator as a minor technique to satisfy the coverage target easily.
 

Our contribution is an implementation of an MOO approach of test case generation for SQL, analysis \& comparison of results and the answer to a suitability of MOO for test case generation for SQL. We successfully implemented the MOO approach. However we concluded that the MOO approach was not applicable for the test case generation of SQL queries, because of its heavy cost of execution time compared to performance.
 
We organized contents as follows. Section 2 describes a genetic algorithm in EvoSQL as a baseline, and the representation of GA setting for SQL test case generation. Section 3 presents our representation of MOO setting, our modified model based on NSGA-II. Section 4 we evaluate our model and analyze failures of it. We conclude the paper in section.
