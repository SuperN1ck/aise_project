\chapter{Problem Formulation}

\section{What is a Coverage Target?}
\label{sec:what_is_a_coverage_target}

To generate test data of SQL queries, the authors of the original paper first decided a test adequacy criterion. Consider the following SQL query:

\begin{verbatim}
SELECT * FROM "Product" WHERE "Type" = 'Cosmetic'
\end{verbatim}
It contains at least two different branches that could be tested:

(1) When a row contains \verb|Type = 'Cosmetic'|.

(2) When a row contains \verb|Type != 'Cosmetic'|.

They selected SQLFpc presented by \cite{de2010constraint}. SQLFpc is a full predicate coverage criterion for SQL queries which considers selection, joining, grouping, aggregate functions, subqueries, case expressions, and null values. Given a SQL query, SQLFpc produces coverage targets in SQL formats. A target is covered when a database returns non-empty results after executing it against the generated data.

As an example, SQLFpc would produce two coverage targets for the query above:
\begin{verbatim}
(1) SELECT * FROM "Product" WHERE ("Type" = 'Cosmetic')
(2) SELECT * FROM "Product" WHERE NOT ("Type" = 'Cosmetic')
\end{verbatim}

EvoSQL gets coverage targets by communicating with SQLFpc web service. The authors of the original paper already provided all extracted coverage targets from the queries because the extraction is done via the SQLFpc web service. Thus, if those targets are not getting pre-stored, one would have to connect SQLFpc all the time. Therefore, once EvoSQL executed and extracted coverage targets with SQLFpc, it would pre-store those coverage targets and serialize them into one file. Next time you execute EvoSQL, it will not extract targets again instead it would reload your file on your local computer. 
The extracted targets provided by the authors of the original paper are in an old serialization version so we re-serialize all roughly 2000 queries again with SQLFpc web service.

\section{Solution Representation}
In section 2.1, we mentioned that a target is covered when a database returns one single row after executing it. That means the database has the test data we want. In EvoSQL, they defined “Fixture” this object is literally a set of tables, where each of them contains a list of rows. Therefore, a candidate solution is a set of tables \textit{T = \{T1, . . . ,Tn \}}, where each table \textit{Ti} is composed of rows, i.e.,\textit{Ti = \{R1, . . . , Rk\}}. Each row contains cells, i.e., \textit{Rj = \{V1, . . . ,Vc \}}, where c is the number of columns in \textit{Ti}. 

The fixture is not a database filled with data so we cannot execute target query to test if it covers the target. They transformed fixture into “INSERT” statements and serialized them. They used HSQLDB, a relational database management system written in Java, first built the table schema and then executed those “INSERT” statements. Therefore, we have a database filled with our test data so now we can execute the coverage target and then got the result whether it covers or not.

\section{Fitness Function}
The authors of the original paper introduced what a query execution plan for a SQL query is. A query execution plan indicates the operations required to process the query and the order by which they need to be performed. As an example in this query, the execution plan consists of two individual steps.
\begin{verbatim}
SELECT * 
FROM Cars
JOIN Tires
ON Cars.tire_id - Tires.id
WHERE model = 'Ferrari'
\end{verbatim}

The first step is that the two tables get joined. The second step is that we filter out some instances. Each step can contain multiple relational algebra operations. In this example, possible step functions are JOINs or WHERE clauses.

Now we have execution plans for SQL queries so we can estimate the fitness of one Fixture. In EvoSQL, they didn't compare two fixtures by a single value. Instead, they measured (1) MaxQueryLevel, which is how many steps can be reached by one fixture (2) step distance, which how far a single fixture is to satisfy the step where the database engine stopped its execution. Therefore, EvoSQL first compared fixtures with MaxQueryLevel. For example, if Fixture A's MaxQueryLevel is bigger than Fixture B, Fixture A is better than Fixture B and vice versa. However, if both MaxQueryLevel is the same, EvoSQL would compare with step distance. If Fixture A's step distance is shorter than Fixture B, Fixture A is better than Fixture B. 
